% Options for packages loaded elsewhere
\PassOptionsToPackage{unicode}{hyperref}
\PassOptionsToPackage{hyphens}{url}
\PassOptionsToPackage{dvipsnames,svgnames,x11names}{xcolor}
%
\documentclass[
  a4paper,
  fontsize=10pt]{scrartcl}

\usepackage{amsmath,amssymb}
\usepackage{iftex}
\ifPDFTeX
  \usepackage[T1]{fontenc}
  \usepackage[utf8]{inputenc}
  \usepackage{textcomp} % provide euro and other symbols
\else % if luatex or xetex
  \usepackage{unicode-math}
  \defaultfontfeatures{Scale=MatchLowercase}
  \defaultfontfeatures[\rmfamily]{Ligatures=TeX,Scale=1}
\fi
\usepackage{lmodern}
\ifPDFTeX\else  
    % xetex/luatex font selection
\fi
% Use upquote if available, for straight quotes in verbatim environments
\IfFileExists{upquote.sty}{\usepackage{upquote}}{}
\IfFileExists{microtype.sty}{% use microtype if available
  \usepackage[]{microtype}
  \UseMicrotypeSet[protrusion]{basicmath} % disable protrusion for tt fonts
}{}
\makeatletter
\@ifundefined{KOMAClassName}{% if non-KOMA class
  \IfFileExists{parskip.sty}{%
    \usepackage{parskip}
  }{% else
    \setlength{\parindent}{0pt}
    \setlength{\parskip}{6pt plus 2pt minus 1pt}}
}{% if KOMA class
  \KOMAoptions{parskip=half}}
\makeatother
\usepackage{xcolor}
\setlength{\emergencystretch}{3em} % prevent overfull lines
\setcounter{secnumdepth}{5}
% Make \paragraph and \subparagraph free-standing
\ifx\paragraph\undefined\else
  \let\oldparagraph\paragraph
  \renewcommand{\paragraph}[1]{\oldparagraph{#1}\mbox{}}
\fi
\ifx\subparagraph\undefined\else
  \let\oldsubparagraph\subparagraph
  \renewcommand{\subparagraph}[1]{\oldsubparagraph{#1}\mbox{}}
\fi


\providecommand{\tightlist}{%
  \setlength{\itemsep}{0pt}\setlength{\parskip}{0pt}}\usepackage{longtable,booktabs,array}
\usepackage{calc} % for calculating minipage widths
% Correct order of tables after \paragraph or \subparagraph
\usepackage{etoolbox}
\makeatletter
\patchcmd\longtable{\par}{\if@noskipsec\mbox{}\fi\par}{}{}
\makeatother
% Allow footnotes in longtable head/foot
\IfFileExists{footnotehyper.sty}{\usepackage{footnotehyper}}{\usepackage{footnote}}
\makesavenoteenv{longtable}
\usepackage{graphicx}
\makeatletter
\def\maxwidth{\ifdim\Gin@nat@width>\linewidth\linewidth\else\Gin@nat@width\fi}
\def\maxheight{\ifdim\Gin@nat@height>\textheight\textheight\else\Gin@nat@height\fi}
\makeatother
% Scale images if necessary, so that they will not overflow the page
% margins by default, and it is still possible to overwrite the defaults
% using explicit options in \includegraphics[width, height, ...]{}
\setkeys{Gin}{width=\maxwidth,height=\maxheight,keepaspectratio}
% Set default figure placement to htbp
\makeatletter
\def\fps@figure{htbp}
\makeatother

\usepackage{siunitx}
\AddToHook{env/tabular/before}{\addfontfeatures{Numbers=Monospaced}}
\AddToHook{env/longtable/before}{\addfontfeatures{Numbers=Monospaced}}
\usepackage{alphabeta}
\usepackage{marginnote}
\usepackage[math=arsenal+kpsans]{arsenal}
\renewcommand*{\marginfont}{\footnotesize}
\KOMAoption{captions}{tableheading}
\makeatletter
\makeatother
\makeatletter
\makeatother
\makeatletter
\@ifpackageloaded{caption}{}{\usepackage{caption}}
\AtBeginDocument{%
\ifdefined\contentsname
  \renewcommand*\contentsname{Table des matières}
\else
  \newcommand\contentsname{Table des matières}
\fi
\ifdefined\listfigurename
  \renewcommand*\listfigurename{Liste des Figures}
\else
  \newcommand\listfigurename{Liste des Figures}
\fi
\ifdefined\listtablename
  \renewcommand*\listtablename{Liste des Tables}
\else
  \newcommand\listtablename{Liste des Tables}
\fi
\ifdefined\figurename
  \renewcommand*\figurename{Figure}
\else
  \newcommand\figurename{Figure}
\fi
\ifdefined\tablename
  \renewcommand*\tablename{Table}
\else
  \newcommand\tablename{Table}
\fi
}
\@ifpackageloaded{float}{}{\usepackage{float}}
\floatstyle{ruled}
\@ifundefined{c@chapter}{\newfloat{codelisting}{h}{lop}}{\newfloat{codelisting}{h}{lop}[chapter]}
\floatname{codelisting}{Listing}
\newcommand*\listoflistings{\listof{codelisting}{Liste des Listings}}
\makeatother
\makeatletter
\@ifpackageloaded{caption}{}{\usepackage{caption}}
\@ifpackageloaded{subcaption}{}{\usepackage{subcaption}}
\makeatother
\makeatletter
\@ifpackageloaded{tcolorbox}{}{\usepackage[skins,breakable]{tcolorbox}}
\makeatother
\makeatletter
\@ifundefined{shadecolor}{\definecolor{shadecolor}{rgb}{.97, .97, .97}}
\makeatother
\makeatletter
\makeatother
\makeatletter
\makeatother
\ifLuaTeX
\usepackage[bidi=basic]{babel}
\else
\usepackage[bidi=default]{babel}
\fi
\babelprovide[main,import]{french}
% get rid of language-specific shorthands (see #6817):
\let\LanguageShortHands\languageshorthands
\def\languageshorthands#1{}
\ifLuaTeX
  \usepackage{selnolig}  % disable illegal ligatures
\fi
\IfFileExists{bookmark.sty}{\usepackage{bookmark}}{\usepackage{hyperref}}
\IfFileExists{xurl.sty}{\usepackage{xurl}}{} % add URL line breaks if available
\urlstyle{same} % disable monospaced font for URLs
\hypersetup{
  pdftitle={latexr\_papier},
  pdfauthor={D Philippe },
  pdflang={fr},
  colorlinks=true,
  linkcolor={blue},
  filecolor={Maroon},
  citecolor={Blue},
  urlcolor={Blue},
  pdfcreator={LaTeX via pandoc}}

\title{latexr\_papier}
\usepackage{etoolbox}
\makeatletter
\providecommand{\subtitle}[1]{% add subtitle to \maketitle
  \apptocmd{\@title}{\par {\large #1 \par}}{}{}
}
\makeatother
\subtitle{Démonstration en \LaTeX}
\author{D\textsuperscript{r} Philippe \textsc{Michel}}
\date{}

\begin{document}
\maketitle
\ifdefined\Shaded\renewenvironment{Shaded}{\begin{tcolorbox}[boxrule=0pt, breakable, sharp corners, interior hidden, borderline west={3pt}{0pt}{shadecolor}, frame hidden, enhanced]}{\end{tcolorbox}}\fi

\renewcommand*\contentsname{Table des matières}
{
\hypersetup{linkcolor=}
\setcounter{tocdepth}{3}
\tableofcontents
}
\listoffigures
\listoftables
\hypertarget{travaux-pruxe9paratoires}{%
\section{Travaux préparatoires}\label{travaux-pruxe9paratoires}}

À peine fus−je réveillé le lendemain que j'allais visiter les dehors du
château, et célébrer mon avènement à la solitude. Le perron faisait face
au nord−ouest. Quand on était assis sur le diazome de ce perron, on
avait devant soi la Cour Verte, et au delà de cette cour, un potager
étendu entre deux futaies : l'une, à droite (le quinconce par lequel
nous étions arrivés), s'appelait le petit Mail ; l'autre, à gauche, le
grand Mail. Celle−ci était un bois de chênes, de hêtres, de sycomores,
d'ormes et de châtaigniers. Madame de Sévigné vantait de son temps ces
vieux ombrages ; depuis cette époque, cent quarante années avaient été
ajoutées à leur beauté.

La moyenne des âges est de 87.9 ans.

\hypertarget{description}{%
\section{Description}\label{description}}

\hypertarget{tableau-principal}{%
\subsection{Tableau principal}\label{tableau-principal}}

\hypertarget{tbl-tab1}{}
\begin{table}[H]
\caption{\label{tbl-tab1}Description de la population }\tabularnewline

\centering
\resizebox{\linewidth}{!}{
\begin{tabular}{lccc}
\toprule
\multicolumn{1}{c}{ } & \multicolumn{2}{c}{\textbf{Escarres}} & \multicolumn{1}{c}{ } \\
\cmidrule(l{3pt}r{3pt}){2-3}
 & \textbf{non}, N = 35 & \textbf{oui}, N = 15 & \textbf{p-value}\\
\midrule
\textbf{Sexe} &  &  & \textbf{0.007}\\
\hspace{1em}f & 19 (54\%) & 2 (13\%) & \\
\hspace{1em}m & 16 (46\%) & 13 (87\%) & \\
\textbf{Âge} & 87.66 ± 2.71 & 88.33 ± 3.09 & 0.5\\
\textbf{Mode d’admission} &  &  & 0.3\\
\addlinespace
\hspace{1em}court séjour & 15 (44\%) & 11 (73\%) & \\
\hspace{1em}smur & 6 (18\%) & 2 (13\%) & \\
\hspace{1em}ssr & 3 (8.8\%) & 0 (0\%) & \\
\hspace{1em}urgences & 10 (29\%) & 2 (13\%) & \\
\textbf{Lieu de vie avant} &  &  & 0.6\\
\addlinespace
\hspace{1em}Avec la famille & 6 (19\%) & 1 (6.7\%) & \\
\hspace{1em}Domicile, seul & 12 (39\%) & 4 (27\%) & \\
\hspace{1em}EHPAD & 2 (6.5\%) & 1 (6.7\%) & \\
\hspace{1em}En couple & 8 (26\%) & 7 (47\%) & \\
\hspace{1em}Maison de retraite & 3 (9.7\%) & 2 (13\%) & \\
\addlinespace
\textbf{IGS 2} & 54 ± 21 & 54 ± 20 & >0.9\\
\textbf{Type d’admission} &  &  & 0.4\\
\hspace{1em}Chirurgie non programmée & 4 (11\%) & 3 (25\%) & \\
\hspace{1em}Chirurgie programmée & 2 (5.7\%) & 1 (8.3\%) & \\
\hspace{1em}Médical & 29 (83\%) & 8 (67\%) & \\
\addlinespace
\textbf{Durée de séjour en réanimation} & 5.5 ± 3.2 & 6.0 ± 6.5 & 0.8\\
\bottomrule
\multicolumn{4}{l}{\rule{0pt}{1em}\textsuperscript{1} n (\%); Mean ± SD}\\
\multicolumn{4}{l}{\rule{0pt}{1em}\textsuperscript{2} Pearson's Chi-squared test; Welch Two Sample t-test; Fisher's exact test}\\
\end{tabular}}
\end{table}

\begin{figure}

{\centering \includegraphics{latexr_papier_files/figure-pdf/fig-age_esc-1.pdf}

}

\caption{\label{fig-age_esc}Prévalences des escarres selon l'âge}

\end{figure}



\end{document}
